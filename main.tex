\documentclass[a4paper,french,final,article]{memoir}
\input{backend_code/preambule}
%\input{backend_code/bibliographie}
\input{personalisation}

\begin{document}
\maketitle
%\tableofcontents
\addtocounter{chapter}{2}
\addtocounter{exercise}{7}
\begin{exercise}[Résultat]\label{ex:barymat}
Soit ABC un triangle \emph{non-dégénéré} de l'espace affine $\mathbf{R}^3$.

\noindent Soient encore 3 points de $R^3$, ${P_1=(t_1,t_2,t_3),P_2=(u_1,u_2,u_3), P_3=(v_1,v_2,v_3)}$,
où les coordonées sont exprimées dans le repère barycentrique $(A,B,C)$, et de somme égale à 1. On a~:
\[\boxed{\text{$P_1,P_2~\&~ P_3$ sont alignés} \iff \begin{vmatrix}
  t_1 & t_2 & t_3 \\ u_1 & u_2 & u_3 \\ v_1 & v_2 & v_3
\end{vmatrix}=0}\] 
\end{exercise}
\begin{exercise}[Résultat]\label{ex:baryfrac}
Soient A et B deux points du plan. Soit aussi deux réels $\alpha$ et $\beta$ tels que $\alpha+\beta=1$. 

\noindent Soit alors $G=\Bary\{(A,\alpha),(B,\beta)\}$. 
On a~: 
\[\boxed{\alpha=\frac{\overline{GB}}{\overline{AB}}~\&~\beta=\frac{\overline{GA}}{\overline{BA}}}\]
\end{exercise}
\begin{exercise}[Théorème de Ménélaüs]
Soit $ABC$ un triangle  \emph{non-dégénéré} et $A',B',C'$ trois points distincts des sommets tels que~: \[A' \in (BC),B'\in (CA), C' \in (AB)\]

\noindent Montrer que ces points sont alignés si et selement si~: 
\[\frac{\overline{A'B}}{\overline{A'C}}\frac{\overline{B'C}}{\overline{B'A}}\frac{\overline{C'A}}{\overline{C'B}}=1\]
\end{exercise}
\begin{proof}
Tout d'abord, comme ABC est non-dégénéré, ses sommets forment un repère barycentrique du plan;Les hypothèses d'appartenance ci-dessus donnent donc l'existence de coefficients $\left(\alpha_i\right)_{i \in \left\lbrace A',B',C'\right\rbrace},\left(\beta_i\right)_{i \in \left\lbrace A',B',C'\right\rbrace},\left(\gamma_i\right)_{i \in \left\lbrace A',B',C'\right\rbrace}$ tels que, les hypothèses de  l' \cref{ex:barymat} soient vérifiées et : 
\[\begin{vmatrix}
  0& \beta_{A'}& \gamma_{A'}\\
  \alpha_{B'} &0& \gamma_{B'}\\
    \alpha_{C'}& \beta_{C'}& 0
\end{vmatrix}=0\]
En développant par rapport à la première colonne, il vient :
\[\alpha_{B'}\beta_{C'}\gamma_{A'}+\gamma_{B'}\alpha_{C'}\beta_{A'}=0\]
C'est-à-dire : 
\[\alpha_{B'}\beta_{C'}\gamma_{A'}= -\left(\gamma_{B'}\alpha_{C'}\beta_{A'}\right)\]
Mais d'après l'\cref{ex:baryfrac}, on a ~:
\[\frac{\overline{A'B}}{\overline{A'C}}=-\frac{\gamma_{A'}}{\beta_{A'}}\]
En appliquant plusieurs fois, on obtient : 
\[\boxed{\frac{\overline{A'B}}{\overline{A'C}}\frac{\overline{B'C}}{\overline{B'A}}\frac{\overline{C'A}}{\overline{C'B}}=\cancel{-}\frac{\cancel{\gamma_{A'}}}{\cancel{\beta_{A'}}}\frac{\cancel{\alpha_{B'}}}{\cancel{\gamma_{A'}}}\left(\cancel{-}\frac{\cancel{\beta_{A'}}}{\cancel{\alpha_{B'}}}\right)=1}\]
\end{proof}
\begin{exercise} [Fonction de \bsc{Leibnitz}]
Soient $(A_j,\lambda_j)_{j \in \left\lBrack 1;n \right\rBrack}$ des points massiques de l'espace euclidien $\mathbb{E}=\mathbb{R}^3$. On définit une application $f~:~\mathbb{E} \to \mathbb{R}$ par $M \mapsto f(M)=\sum_{j=1}^n \lambda_j\norm*{\overrightarrow{MA_j}}^2$ 
\begin{enumerate}
  \item Si $\sum_{i=1}^n \lambda_i=0$, montrer que $\exists V \in \mathbb{E},\forall (M,M')\in \mathbb{E}^2\!,\/ f(M)=f(M')+2\overrightarrow{MM'}\cdot \overrightarrow{V}$. On exprimera $\overrightarrow{V}\!$. 
  \item Si $\sum_{i=1}^n \lambda_i\neq 0$, montrer que~: $$ f(M)=\sum_{j=1}^n \lambda_j\norm*{\overrightarrow{GA_j}}^2+\sum_{j=1}^n \lambda_j\norm*{\overrightarrow{MG}}^2,$$ où $(G, \sum_{i=1}^n \lambda_i)$ est le barycentre des points $(A_j,\lambda_j)_{j \in \left\lBrack 1;n \right\rBrack}$
\end{enumerate}
\end{exercise}
Pour simplifier la composition, on notera aussi $\left\langle u\,;v\right\rangle$ au lieu de $u\cdot v$. On rappelle l'identité du carré scalaire : $$\boxed{\forall (\overrightarrow{AB};\overrightarrow{CD})\in \overrightarrow{\mathbb{E}}^2, \norm*{\overrightarrow{AB}+\overrightarrow{CD}}^2\eqdef \left\langle\overrightarrow{AB}+\overrightarrow{CD}\,;\overrightarrow{AB}+\overrightarrow{CD}\right\rangle=\norm*{\overrightarrow{AB}}^2+\norm*{\overrightarrow{CD}}^2+2\left\langle\overrightarrow{AB}\,;\overrightarrow{CD}\right\rangle}$$
\begin{enumerate}
  \item D'après l'énoncé et la relation de Chasles, on a : 
  \[\overrightarrow{MA_j} =\overrightarrow{MM'}+\overrightarrow{M'A_j}\]
Ainsi, en réinjectant dans $f$, on obtient~:
\begin{align*}
f(M)& =\sum_{j=1}^n \lambda_j\norm*{\overrightarrow{MM'}+\overrightarrow{M'A_j}}^2\\ 
& =\sum_{j=1}^n \lambda_j\left( \norm*{\overrightarrow{MM'}}^2+\norm*{\overrightarrow{M'A_j}}^2+2\left\langle\overrightarrow{MM'}\,;\overrightarrow{M'A_j}\right\rangle\right)\\  
& =\underbrace{\sum_{j=1}^n \lambda_j\norm*{\overrightarrow{M'A_j}}^2}_{f(M')}+\left\langle 2 \overrightarrow{MM'}\,;\sum_{j=1}^n\overrightarrow{M'A_j}\right\rangle +\underbrace{\sum_{j=1}^n \lambda_j \norm*{\overrightarrow{M'M}}^2}_{\vec{0} \text{ car }\sum_{i=1}^n \lambda_i=0}\numbereq \label{eq:exprf}\\
\intertext{ le vecteur $V=\sum_{j=1}^n\overrightarrow{M'A_j}$ est constant par rapport à M et à $i$, d'où~:}
\Aboxed{& \forall (M,M') \in \mathbb{E}^2, f(M)=f(M')+2\overrightarrow{MM'}\cdot \overrightarrow{V}}\tag*{\bsc{c.q.f.d}}
\intertext{\item Si on suppose $\sum_{i=1}^n \lambda_i \neq 0$, Alors le point G barycentre des point (cf énoncé) est bien défini. En reprenant l'\cref{eq:exprf},en posant $M'=G\in \mathbb{E}$ on obtient} 
f(M)& =f(G)+\left\langle 2 \overrightarrow{MM'}\,;\underbrace{\sum_{j=1}^n\overrightarrow{GA_j}}_{\vec{0} \text{ par déf. de G.}}\right\rangle +\sum_{j=1}^n \lambda_j \norm*{\overrightarrow{GM}}^2
\shortintertext{Soit :}
\Aboxed{& f(M)=f(G)+\sum_{j=1}^n \lambda_j \norm*{\overrightarrow{GM}}^2}
 \end{align*}
 \item Par manque de temps, esquisses de réponses.
 \begin{enumerate} [label=(\alph*)]
   \item $\lambda_A$ et $\lambda_B$ sont les coefficients tels que C soit barycentre de A et de B, c'est-à-dire les coordonées barycentriques de C dans $(A;B)$.
   \item Il suffit d'utiliser le point précédent, et de réarranger la somme, en utilisant l'\cref{ex:baryfrac}.
 \end{enumerate}
  \end{enumerate}
\end{document}
