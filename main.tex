\documentclass[a4paper,french,final,article]{memoir}
\usepackage{luacode}
\usepackage{babel}
\usepackage{microtype}
\usepackage{fontspec}
\usepackage{geometry}
\usepackage{mathtools}
\usepackage[math-style=french,warnings-off={mathtools-colon,mathtools-overbracket}]{unicode-math}
\setmainfont{TeX Gyre Termes}
\setmathfont{TeX Gyre Termes Math}
\setmathfont[range=bb]{XITS Math}
\setmathfont[range={scr,cal}]{TeX Gyre Pagella Math}
\usepackage[amsmath,thmmarks,hyperref]{ntheorem}
\usepackage[most]{tcolorbox}
\usepackage{xparse,xpatch}
\usepackage{tikz}
\usepackage{caption}
\usetikzlibrary{calc,trees,positioning,arrows,fit,shapes,calc}
\usepackage{lualatex-math}
\usepackage{crossreftools}
\usepackage[unicode,naturalnames]{hyperref}
\usepackage{cleveref}
\AtBeginDocument{\let\savedemptyset\emptyset} % 
\AtBeginDocument{\let\emptyset\varnothing} % 
\AtBeginDocument{\let\savedleq\leq} % 
\AtBeginDocument{\let\savedgeq\geq} %
\AtBeginDocument{\let\leq\leqslant} % 
\AtBeginDocument{\let\geq\geqslant} %
\newcommand{\paral}{\mathrel{\!/\mkern-5mu/\!}} % \parallel existe déjà : || vs //
\makeatletter
\newcommand*{\theauthor}{\@author}
\newcommand*{\thetitle}{\@title}
\newcommand*{\thedate}{\@date}
\renewcommand{\xmapsto}[2][]{\mathrel{\mathpalette\xmapsto@{{#1}{#2}}}}
\newcommand{\xmapsto@}[2]{\xmapsto@@{#1}#2}
\newcommand{\xmapsto@@}[3]{%
  \begingroup
  \sbox\z@{$\m@th#1\mathop{}\limits_{\;#2\;}^{\;#3\;}$}%
  \mathop{\Uhextensible width \wd\z@ 0 "27FC}_{#2}^{#3}%
  \endgroup
}
\newcounter{proofpart}
\newcommand{\proofpart}{\@ifstar{\@proofpart}{\@@proofpart}}

\newcommand{\@proofpart}[1]{%
  \if\detokenize{#1}\relax\else{
  \par
  \addvspace{\medskipamount}%
  \noindent \itshape%
  {#1~:\par\nobreak\smallskip}%
  \normalfont
  \@afterheading
}
}

\newcommand{\@@proofpart}[1]{%
  \par
  \addvspace{\medskipamount}%
  \stepcounter{proofpart}%
  \noindent Partie \theproofpart~:~\itshape%
  \if\detokenize{#1}\relax%
  \else{#1.}\fi%
  \par\nobreak\smallskip
  \normalfont
  \@afterheading
}

\makeatother
%%%%%%%%%%%%%%%%%%%%%%%%%%%%%%%%%%%%%%%%%%%%%%%%
%            NUMEROTATION (CLASSE MEMOIR)       %
%%%%%%%%%%%%%%%%%%%%%%%%%%%%%%%%%%%%%%%%%%%%%%%%%
\setsecnumdepth{subsubsection}
\renewcommand{\cftpartaftersnum}{.}
\renewcommand{\cftchapteraftersnum}{.}
\renewcommand{\cftpartdotsep}{\cftdotsep}
\renewcommand{\cftchapterdotsep}{\cftdotsep}% Chapters should use dots in ToC
%\aliaspagestyle{title}{empty} 
\aliaspagestyle{part}{empty} 
\setlength\headheight{\dimexpr \headheight+0.20004pt}

%\usepackage[autostyle=true,french=guillemets,maxlevel=3]{csquotes}
\usepackage[backend=biber]{biblatex}
\addbibresource{references.bib}
\BeforeClearDocument{\nocite{*}\printbibliography}
\usepackage{cancel}
\usepackage{enumitem}
%%%%%%%%%%%%%%%%%%%%%%%%%%%%%
%   THEOREMES SANS BOITES   %
%%%%%%%%%%%%%%%%%%%%%%%%%%%%%
\makeatletter
\newtheoremstyle{exo}%
  {\item[\rlap{\vbox{\hbox{\hskip\labelsep \theorem@headerfont
          ##1\ ##2\theorem@separator}\hbox{\strut}}}]}%
  {\item[\rlap{\vbox{\hbox{\hskip\labelsep \theorem@headerfont
%           ##1\ ##2\ (##3)\theorem@separator}\hbox{\strut}}}]}
          ##1\ ##2\ \textnormal{--}\ ##3\theorem@separator}\hbox{\strut}\addvspace{\parskip}}}]}
\makeatother
\theoremstyle{break}
\theoremseparator{~:} % espace fine insécable avant le :
\newtheorem{lemma}{Lemme}
\newtheorem{corollary}{Corollaire}
\newtheorem{definition}{Définition}
\theoremseparator{~:} % espace fine insécable avant le :
\theorembodyfont{\normalfont}
\theoremstyle{exo}
\theoremseparator{} % espace fine insécable avant le :
\newtheorem{exercise}{Exercice}[chapter]
\theoremstyle{plain}
\theorembodyfont{\normalfont}
\newtheorem*{question}{Question}
\newtheorem*{answer}{Réponse}
\newtheorem{remark}{Remarque}
\theoremsymbol{\text{\bsc{c.q.f.d}}} % mod
\theorembodyfont{\normalfont}
\theoremprework{\setcounter{proofpart}{0}}
\newtheorem*{proof}{Démonstration}
%%%%%%%%%%%%%%%%%%%%%%%%%%%%%
%            COULEURS       %
%%%%%%%%%%%%%%%%%%%%%%%%%%%%%
\definecolor{vert}{RGB}{0,181,0}
\definecolor{oran}{RGB}{223,74,0}
\definecolor{viol}{RGB}{134,0,175}
\definecolor{roug}{RGB}{215,15,0}
\definecolor{bleu}{RGB}{0,104,180}

%%%%%%%%%%%%%%%%%%%%%%%%%%%%%
%   BOITES POUR THEOREMES   %
%%%%%%%%%%%%%%%%%%%%%%%%%%%%%
\tcbset{separator sign={},
        description delimiters parenthesis,
        label separator=-,
styletheorem/.style={enhanced,
  coltitle=black,
  colback=white,
  fonttitle=\bfseries,
  boxrule=\fboxrule,
  boxed title style={boxrule=\fboxrule},
  attach boxed title to top left={yshift=-2mm, xshift=2mm},
    }%
}
\newtcbtheorem[auto counter, number within = section]
{theoremb}{Théorème}{styletheorem,colframe=roug,
colback=white!90!roug,colbacktitle=white!80!roug,label type=theorem}{thm}
\newtcbtheorem[auto counter, number within = section]
{remarkb}{Remarque}{styletheorem,colframe=oran,
colbacktitle=white!80!oran,colback=white!90!oran,label type=remark}{rem}
\newtcbtheorem[auto counter, number within = section]
{defb}{Définition}{styletheorem,colframe=bleu,
colbacktitle=white!80!bleu,colback=white!90!bleu,label type=definition}{def}
\newtcbtheorem[auto counter, number within = section]
{noteb}{Commentaire}{styletheorem,colframe=vert,
colbacktitle=white!80!vert,colback=white!90!vert,label type=note}{note}
% le label type fait automatiquement la jonction avec cleveref pour nommer les théorèmes : le dernier groupe (thm,rem,def) permet de créer des labels automatiquement. 

%%%%%%%%%%%%%%%%%%%%%%%%%%%%%%%%%%
%  SEPARATEUR (DANS LES PREUVES) : \proofpart   %
%%%%%%%%%%%%%%%%%%%%%%%%%%%%%%%%%%%
% une  commande est définie pour séparer les preuves en deux variantes : avec et sans étoiles. (\proofpart et \proofpart*)
% Par défaut, elle crée des sous parties dans un environement de démonstration sous la forme 
% " Partie n : [titre en italique]. ", où n est un entier naturel strictement positif. Dans le cas ou le titre est vide, le point (.) n'est pas ajouté. Si le titre est vide, il faut utiliser \proofpart{}
% Avec une étoile, on obtient 
%[titre en italique : ] 

\newcommand{\deffunct}[5]{%
\begin{align*}%
      #1 \colon & #2 \to #3\\
       &#4\xmapsto{\hphantom{#1}} #5
\end{align*}%
}
%%%%%%%%%%%%%%%%%%%%%%%%%%%%%
%   PAGE DE GARDE            %
%%%%%%%%%%%%%%%%%%%%%%%%%%%%%
\newcommand{\HRule}{\rule{\paperwidth}{0.5mm}} % trait, régler eppaisseur
\newcommand*{\theuniversity}{Université de Toulon}
\newcommand*{\theyearname}{Licence de Mathématiques, parcours mathématiques, 2\ieme~année}
\newcommand*{\thesupervisor}{Cours dispensés par Joachim \bsc{Asch} \par Travaux dirigés assurés par Stéphane \bsc{Leblanc}}
\AtBeginDocument{\author{Tom \bsc{Domenge}}}
\AtBeginDocument{\title{Devoir Maison de géométrie\par \theyearname \par Année universitaire 2020-21}}
\AtBeginDocument{\date{\`A rendre pour le 12 février 2021}}
%%%%%%%%%%%%%%%%%%%%%%%%%%%%%%%%%%%%%%%%%%%%%%
%   Opérateurs (dans le style de max etc.   %
%%%%%%%%%%%%%%%%%%%%%%%%%%%%%%%%%%%%%%%%%%%%%%
\DeclareMathOperator{\Card}{Card} % s'utilise avec \Card en mode maths
\DeclareMathOperator{\Bary}{bar}
\newcommand*{\numbereq}{\stepcounter{equation}\tag{\theequation}}
\DeclarePairedDelimiter{\norm}{\lVert}{\rVert}


\begin{document}
\maketitle
%\tableofcontents
\addtocounter{chapter}{2}
\addtocounter{exercise}{7}
\begin{exercise}[Résultat]\label{ex:barymat}
Soit ABC un triangle \emph{non-dégénéré} de l'espace affine $\mathbf{R}^3$.

\noindent Soient encore 3 points de $R^3$, ${P_1=(t_1,t_2,t_3),P_2=(u_1,u_2,u_3), P_3=(v_1,v_2,v_3)}$,
où les coordonées sont exprimées dans le repère barycentrique $(A,B,C)$, et de somme égale à 1. On a~:
\[\boxed{\text{$P_1,P_2~\&~ P_3$ sont alignés} \iff \begin{vmatrix}
  t_1 & t_2 & t_3 \\ u_1 & u_2 & u_3 \\ v_1 & v_2 & v_3
\end{vmatrix}=0}\] 
\end{exercise}
\begin{exercise}[Résultat]\label{ex:baryfrac}
Soient A et B deux points du plan. Soit aussi deux réels $\alpha$ et $\beta$ tels que $\alpha+\beta=1$. 

\noindent Soit alors $G=\Bary\{(A,\alpha),(B,\beta)\}$. 
On a~: 
\[\boxed{\alpha=\frac{\overline{GB}}{\overline{AB}}~\&~\beta=\frac{\overline{GA}}{\overline{BA}}}\]
\end{exercise}
\begin{exercise}[Théorème de Ménélaüs]
Soit $ABC$ un triangle  \emph{non-dégénéré} et $A',B',C'$ trois points distincts des sommets tels que~: \[A' \in (BC),B'\in (CA), C' \in (AB)\]

\noindent Montrer que ces points sont alignés si et selement si~: 
\[\frac{\overline{A'B}}{\overline{A'C}}\frac{\overline{B'C}}{\overline{B'A}}\frac{\overline{C'A}}{\overline{C'B}}=1\]
\end{exercise}
\begin{proof}
Tout d'abord, comme ABC est non-dégénéré, ses sommets forment un repère barycentrique du plan;Les hypothèses d'appartenance ci-dessus donnent donc l'existence de coefficients $\left(\alpha_i\right)_{i \in \left\lbrace A',B',C'\right\rbrace},\left(\beta_i\right)_{i \in \left\lbrace A',B',C'\right\rbrace},\left(\gamma_i\right)_{i \in \left\lbrace A',B',C'\right\rbrace}$ tels que, les hypothèses de  l' \cref{ex:barymat} soient vérifiées et : 
\[\begin{vmatrix}
  0& \beta_{A'}& \gamma_{A'}\\
  \alpha_{B'} &0& \gamma_{B'}\\
    \alpha_{C'}& \beta_{C'}& 0
\end{vmatrix}=0\]
En développant par rapport à la première colonne, il vient :
\[\alpha_{B'}\beta_{C'}\gamma_{A'}+\gamma_{B'}\alpha_{C'}\beta_{A'}=0\]
C'est-à-dire : 
\[\alpha_{B'}\beta_{C'}\gamma_{A'}= -\left(\gamma_{B'}\alpha_{C'}\beta_{A'}\right)\]
Mais d'après l'\cref{ex:baryfrac}, on a ~:
\[\frac{\overline{A'B}}{\overline{A'C}}=-\frac{\gamma_{A'}}{\beta_{A'}}\]
En appliquant plusieurs fois, on obtient : 
\[\boxed{\frac{\overline{A'B}}{\overline{A'C}}\frac{\overline{B'C}}{\overline{B'A}}\frac{\overline{C'A}}{\overline{C'B}}=\cancel{-}\frac{\cancel{\gamma_{A'}}}{\cancel{\beta_{A'}}}\frac{\cancel{\alpha_{B'}}}{\cancel{\gamma_{A'}}}\left(\cancel{-}\frac{\cancel{\beta_{A'}}}{\cancel{\alpha_{B'}}}\right)=1}\]
\end{proof}
\begin{exercise} [Fonction de \bsc{Leibnitz}]
Soient $(A_j,\lambda_j)_{j \in \left\lBrack 1;n \right\rBrack}$ des points massiques de l'espace euclidien $\mathbb{E}=\mathbb{R}^3$. On définit une application $f~:~\mathbb{E} \to \mathbb{R}$ par $M \mapsto f(M)=\sum_{j=1}^n \lambda_j\norm*{\overrightarrow{MA_j}}^2$ 
\begin{enumerate}
  \item Si $\sum_{i=1}^n \lambda_i=0$, montrer que $\exists V \in \mathbb{E},\forall (M,M')\in \mathbb{E}^2\!,\/ f(M)=f(M')+2\overrightarrow{MM'}\cdot \overrightarrow{V}$. On exprimera $\overrightarrow{V}\!$. 
  \item Si $\sum_{i=1}^n \lambda_i\neq 0$, montrer que~: $$ f(M)=\sum_{j=1}^n \lambda_j\norm*{\overrightarrow{GA_j}}^2+\sum_{j=1}^n \lambda_j\norm*{\overrightarrow{MG}}^2,$$ où $(G, \sum_{i=1}^n \lambda_i)$ est le barycentre des points $(A_j,\lambda_j)_{j \in \left\lBrack 1;n \right\rBrack}$
\end{enumerate}
\end{exercise}
Pour simplifier la composition, on notera aussi $\left\langle u\,;v\right\rangle$ au lieu de $u\cdot v$. On rappelle l'identité du carré scalaire : $$\boxed{\forall (\overrightarrow{AB};\overrightarrow{CD})\in \overrightarrow{\mathbb{E}}^2, \norm*{\overrightarrow{AB}+\overrightarrow{CD}}^2\eqdef \left\langle\overrightarrow{AB}+\overrightarrow{CD}\,;\overrightarrow{AB}+\overrightarrow{CD}\right\rangle=\norm*{\overrightarrow{AB}}^2+\norm*{\overrightarrow{CD}}^2+2\left\langle\overrightarrow{AB}\,;\overrightarrow{CD}\right\rangle}$$
\begin{enumerate}
  \item D'après l'énoncé et la relation de Chasles, on a : 
  \[\overrightarrow{MA_j} =\overrightarrow{MM'}+\overrightarrow{M'A_j}\]
Ainsi, en réinjectant dans $f$, on obtient~:
\begin{align*}
f(M)& =\sum_{j=1}^n \lambda_j\norm*{\overrightarrow{MM'}+\overrightarrow{M'A_j}}^2\\ 
& =\sum_{j=1}^n \lambda_j\left( \norm*{\overrightarrow{MM'}}^2+\norm*{\overrightarrow{M'A_j}}^2+2\left\langle\overrightarrow{MM'}\,;\overrightarrow{M'A_j}\right\rangle\right)\\  
& =\underbrace{\sum_{j=1}^n \lambda_j\norm*{\overrightarrow{M'A_j}}^2}_{f(M')}+\left\langle 2 \overrightarrow{MM'}\,;\sum_{j=1}^n\overrightarrow{M'A_j}\right\rangle +\underbrace{\sum_{j=1}^n \lambda_j \norm*{\overrightarrow{M'M}}^2}_{\vec{0} \text{ car }\sum_{i=1}^n \lambda_i=0}\numbereq \label{eq:exprf}\\
\intertext{ le vecteur $V=\sum_{j=1}^n\overrightarrow{M'A_j}$ est constant par rapport à M et à $i$, d'où~:}
\Aboxed{& \forall (M,M') \in \mathbb{E}^2, f(M)=f(M')+2\overrightarrow{MM'}\cdot \overrightarrow{V}}\tag*{\bsc{c.q.f.d}}
\intertext{\item Si on suppose $\sum_{i=1}^n \lambda_i \neq 0$, Alors le point G barycentre des point (cf énoncé) est bien défini. En reprenant l'\cref{eq:exprf},en posant $M'=G\in \mathbb{E}$ on obtient} 
f(M)& =f(G)+\left\langle 2 \overrightarrow{MM'}\,;\underbrace{\sum_{j=1}^n\overrightarrow{GA_j}}_{\vec{0} \text{ par déf. de G.}}\right\rangle +\sum_{j=1}^n \lambda_j \norm*{\overrightarrow{GM}}^2
\shortintertext{Soit :}
\Aboxed{& f(M)=f(G)+\sum_{j=1}^n \lambda_j \norm*{\overrightarrow{GM}}^2}
 \end{align*}
 \item Par manque de temps, esquisses de réponses.
 \begin{enumerate} [label=(\alph*)]
   \item $\lambda_A$ et $\lambda_B$ sont les coefficients tels que C soit barycentre de A et de B, c'est-à-dire les coordonées barycentriques de C dans $(A;B)$.
   \item Il suffit d'utiliser le point précédent, et de réarranger la somme, en utilisant l'\cref{ex:baryfrac}.
 \end{enumerate}
  \end{enumerate}
\end{document}
