\documentclass[a4paper,french,final,article]{memoir}
\input{backend_code/preambule}
%\input{backend_code/bibliographie}
\input{personalisation}
\begin{document}
\begin{titlingpage}
\input{backend_code/page_de_garde}
\end{titlingpage}
\frontmatter
%\tableofcontents
\mainmatter
\addtocounter{chapter}{2}
\addtocounter{exercise}{2}
\begin{exercise}
\'Enoncez et démontrez la propriété d'associativité du barycentre.
\end{exercise}
 
\begin{proof}
	Sans perte de généralité, on démontre la propriété dans le cas de 3 points du plan affine. Soient $a,b,c$ trois réels, tels que $a+b\neq 0,\ a+b+c\neq 0$; $A$, $B$, $C$, trois points de l'espace. On consdière le système pondéré~$\left\lbrace(A,a),(B,b),(C,c)\right\rbrace$.
On appelle alors $G$ le barycentre de $A$, $B$, $C$; $A'$, celui de $A$~et~$B$ dans ce système, tous deux bien définis d'après ce qui précède.

\noindent On veut démontrer l'équivalence :   
\begin{align*}
\Aboxed{G=\Bary\left\lbrace(A,a),(B,b),(C,c)\right\rbrace\iff & G=\Bary\left\lbrace(A',a+b),(C,c)\right\rbrace}
\intertext{par définition de $A'$, on a : }
\iff a\overrightarrow{A'A}+b\overrightarrow{A'B}& =\vec{0}\\
\intertext{par définition de $G$, on a : }
 a\,\overrightarrow{GA}+b\,\overrightarrow{GB}+c\overrightarrow{GC}& =\vec{0}\\
\intertext{On cherche à introduire $A'\!\!$. D'après la relation de Chasles, on obtient :}
 \iff a\left(\overrightarrow{GA'}+\overrightarrow{A'A}\right)+b\left(\overrightarrow{GA'}+\overrightarrow{A'B}\right)+c\overrightarrow{GC}& =\vec{0}\\
\intertext{soit, après regroupement des termes pour faire apparaître le barycentre de A et B : }
\iff (a+b)\overrightarrow{GA'}+\underbrace{a\overrightarrow{A'A}+b\overrightarrow{A'B}}_{\,=\,\vec{0}\text{ par définition de }A'}+\,c\overrightarrow{GC}& =\vec{0}\\
\shortintertext{On a donc obtenu :}
\iff\;\Aboxed{(a+b)\overrightarrow{GA'}+c\overrightarrow{GC}& =\vec{0}}
\end{align*}
Et G est le barycentre de $\left\lbrace(A',a+b),(C,c)\right\rbrace$, par définition. 
\end{proof}
\end{document}
